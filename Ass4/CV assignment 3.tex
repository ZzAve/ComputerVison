%%%%%%%%%%%%%%%%%%%%%%%%%%%%%%%%%%%%%%%%%
% Short Sectioned Assignment
% LaTeX Template
% Version 1.0 (5/5/12)
%
% This template has been downloaded from:
% http://www.LaTeXTemplates.com
%
% Original author:
% Frits Wenneker (http://www.howtotex.com)
%
% License:
% CC BY-NC-SA 3.0 (http://creativecommons.org/licenses/by-nc-sa/3.0/)
%
%%%%%%%%%%%%%%%%%%%%%%%%%%%%%%%%%%%%%%%%%

%----------------------------------------------------------------------------------------
%	PACKAGES AND OTHER DOCUMENT CONFIGURATIONS
%----------------------------------------------------------------------------------------

\documentclass[paper=a4, fontsize=11pt]{scrartcl} % A4 paper and 11pt font size

\usepackage[T1]{fontenc} % Use 8-bit encoding that has 256 glyphs
\usepackage{fourier} % Use the Adobe Utopia font for the document - comment this line to return to the LaTeX default
\usepackage[english, dutch]{babel} % English language/hyphenation
\usepackage{amsmath,amsfonts,amsthm} % Math packages

\usepackage{lipsum} % Used for inserting dummy 'Lorem ipsum' text into the template

\usepackage{sectsty} % Allows customizing section commands
\allsectionsfont{\centering \normalfont\scshape} % Make all sections centered, the default font and small caps

\usepackage{fancyhdr} % Custom headers and footers
\pagestyle{fancyplain} % Makes all pages in the document conform to the custom headers and footers
\fancyhead{} % No page header - if you want one, create it in the same way as the footers below
\fancyfoot[L]{} % Empty left footer
\fancyfoot[C]{} % Empty center footer
\fancyfoot[R]{\thepage} % Page numbering for right footer
\renewcommand{\headrulewidth}{0pt} % Remove header underlines
\renewcommand{\footrulewidth}{0pt} % Remove footer underlines
\setlength{\headheight}{13.6pt} % Customize the height of the header

\numberwithin{equation}{section} % Number equations within sections (i.e. 1.1, 1.2, 2.1, 2.2 instead of 1, 2, 3, 4)
\numberwithin{figure}{section} % Number figures within sections (i.e. 1.1, 1.2, 2.1, 2.2 instead of 1, 2, 3, 4)
\numberwithin{table}{section} % Number tables within sections (i.e. 1.1, 1.2, 2.1, 2.2 instead of 1, 2, 3, 4)

%\setlength\parindent{0pt} % Removes all indentation from paragraphs - comment this line for an assignment with lots of text

%----------------------------------------------------------------------------------------
%	TITLE SECTION
%----------------------------------------------------------------------------------------

\newcommand{\horrule}[1]{\rule{\linewidth}{#1}} % Create horizontal rule command with 1 argument of height

\title{	
\normalfont \normalsize 
\textsc{Universiteit Utrecht, Department of Information and Computing Sciences} \\ [25pt] % Your university, school and/or department name(s)
\horrule{0.5pt} \\[0.4cm] % Thin top horizontal rule
\huge Assignment 4 - Face detection \\ % The assignment title
\horrule{2pt} \\[0.5cm] % Thick bottom horizontal rule
}

\author{Julius van Dis 4038010, Nico Naus, 3472353} % Your name

\date{\normalsize\today} % Today's date or a custom date

\begin{document}

\maketitle % Print the title

%----------------------------------------------------------------------------------------
%	PROBLEM 1
%----------------------------------------------------------------------------------------
\section*{4.a.i. Explain what is linear about the linear Support Vector Machine.}
The linear support vector machine uses a linear equation to divide the items in two classes. This can be in more than 2 dimensions. See the example below.

y = ax +b
z = ax +by + c
...
\section*{4.b.iii. What is the initial validation score of the model?}
The validation score of the initial model is 96.4844\%
\section*{4.c.i. What does this parameter do?}
C is the penalty factor. This is a penalty for non-seperateble points. Choosing the right C-value prevents over or underfitting.
\section*{4.c.ii. Find the optimal C-value by trying different settings and different amounts of learning data.}
We found that the following settings gives the best validation score:\\
Positive images: 200\\
Factor: 1\\
C: 0.001 - 0.1\\
Validation score:97.25\% \\
\section*{4.c.ii. What, in terms of the theory of SVM, does the model consist of? Why does that look like a face?}
The model consists of a 2d vector, dividing the data in two classes, a face or not a face. This conceptually means that if it looks more like a "face" than the model, it gets labelled "face", if its less, "not a face". That is also the reason why it looks like a face.
\section*{4.c.iv. What do these switches do with the learning data? Why is this good for the performance?}
Whitening decorrelates the data. Usually there is bias in the color of the images. Whitening removes bias and correlation, and makes the faces look more alike.\\
Equalizing \\
While testing different settings, we found that whitening didnt have a positive effect on the score. Equalizing also didn't improve the score, but we found that it gave better results. When equalizing is turned on, a C-value of 0.006 gives a score of 95\% \\
\section*{5.c Report the impact on the performance of the real images.}
\section*{6.d Report the impact on the performance of the validation data.}
\section*{6.e Report the impact on the performance of the real images.}
\section*{6.f You get bonus points if you implement another image descriptor and report its performance.}
\section*{7.b Put the Precision/Recall curve of your best performing model for img1.jpg in your report (or add an image of the graph to your archive).}
\section*{7.c Calculate the AP of your best model for img1.jpg.}
\end{document}
